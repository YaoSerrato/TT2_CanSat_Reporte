\chapter*{Objetivos}

\section*{Objetivo general}
%\blindtext
Dise\~nar y construir un sat\'elite tipo CanSat (contenedor "-- veh\'iculo cient\'ifico) con sistema de descenso por paraca\'idas (para el contenedor) y por autorrotaci\'on (para el veh\'iculo cient\'ifico) que transmita durante el vuelo datos de telemetr\'ia a una estaci\'on en Tierra para su posterior an\'alisis, y que tenga una c\'amara apuntando en una sola direcci\'on para grabar toda la etapa de descenso.

\section*{Objetivos Particulares.}
%\blindlist{7}
%\blinditemize,
%\blindenumerate
\begin{itemize}
\item[$\bullet$]Dise\~nar y construir la estructura mec\'anica del contenedor y del veh\'iculo cient\'ifico para soportar valores de aceleraci\'on de hasta 30 [Gs], con el prop\'osito de que el CanSat sea capaz de mantener su integridad f\'isica bajo la acci\'on de cualquier fuerza durante las etapas de despegue, descenso y aterrizaje.

\item[$\bullet$]Dise\~nar y construir el sistema de descenso por paraca\'idas del CanSat para alcanzar una velocidad de descenso de 20 [m/s] $\pm$ 5 [m/s] antes de llegar a los 450 [m] de altura.

\item[$\bullet$]Dise\~nar y construir el sistema de descenso por autorrotaci\'on del veh\'iculo cient\'ifico para alcanzar una velocidad de descenso comprendida dentro del rango de 10 [m/s] a 15 [m/s], para reducir la fuerza de impacto del veh\'iculo cient\'ifico con el suelo al aterrizar.

\item[$\bullet$]Dise\~nar y construir los mecanismos correspondientes para la separaci\'on del veh\'iculo cient\'ifico del contendor, para el despliegue del sistema de descenso por autorrotaci\'on, y para la apertura del paraca\'idas.

\item[$\bullet$]Dise\~nar y construir un mecanismo de orientaci\'on que mantenga una c\'amara apuntando a $45^{\circ}$ con respecto del nadir, y que est\'e dirigida en una sola direcci\'on con respecto del campo magn\'etico de la Tierra dentro de una tolerancia de $\pm$ $10^{\circ}$. 

\item[$\bullet$]Dise\~nar e implementar el sistema electr\'onico para la adquisici\'on, procesamiento, almacenamiento, y env\'io de datos de telemetr\'ia a la estaci\'on en Tierra, y para la ejecuci\'on de las operaciones de control de descenso y de orientaci\'on de c\'amara.

\item[$\bullet$]Dise\~nar e implementar la estaci\'on en Tierra para la recepci\'on, procesamiento, despliegue en pantalla y almacenamiento de los datos de telemetr\'ia enviados por el veh\'iculo cient\'ifico desde el despegue hasta el aterrizaje.

\end{itemize}