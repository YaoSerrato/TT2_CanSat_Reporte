\chapter{Resumen}

El presente trabajo muestra una soluci�n al problema de dise�ar un CanSat con un sistema de descenso por autorrotaci�n. Esta idea surge del concurso internacional \textit{Annual CanSat Competition}, llevada a cabo en Texas, EUA. La soluci�n presentada en este documento est� guiada por las especificaciones dadas por la competencia.\\

\noindent Se comenzar� abordando los temas relacionados al dise�o preliminar del sistema, haciendo uso de diferentes herramientas de dise�o que poseen un enfoque funcional.\\

\noindent Posteriormente, se hablar� sobre la importancia de definir funciones basadas en requerimientos, para despu�s mostrar la jerarqu�a y composici�n de los sistemas y subsistemas del CanSat. Por �ltimo, en esta primera secci�n, se mostrar� el concepto final del CanSat.\\

\noindent Seguido de esto, se entrar� de lleno en el dise�o detallado del sistema, mostrando los criterios, m�todos y herramientas usados para la concepci�n de cada sistema de manera individual. Ser�n abordados los sistemas electr�nico y de gesti�n de informaci�n, para posteriormente hablar sobre los sistemas estructural y de energ�a, as� como los sistemas propios del CanSat.\\

\noindent Posteriormente, se mostrar�n las pruebas de integraci�n y verificaci�n de los sistemas y subsistemas, pudiendo as� apreciar los resultados obtenidos en cada prueba. Estas pruebas mencionan los requerimientos que buscan verificar, el procedimiento que se us�, los resultados esperados y los resultados obtenidos.\\

\noindent Tambi�n se abordar� el an�lisis de los resultados obtenidos durante las pruebas hechas al CanSat en su totalidad. Estas pruebas corresponden a los lanzamientos realizados con un dron y a una altura mucho menor de la especificada en la competencia en su edici�n 2019. Aqu� se podr�n observar las gr�ficas recopiladas durante los lanzamientos.\\

\noindent Finalmente, se presentar�n las conclusiones.\\

