\chapter{Abstract}

The present work shows a solution to the problem of designing a CanSat with an autorotation descent system. This idea arises from the international Annual CanSat Competition \cite{CanSat-competition-website}, held in the United States. The solution presented in this document is guided by the requirements given by the competition\\

\noindent The document will begin by addressing the issues related to the preliminary design of the system, making use of different design tools that have a functional approach. Subsequently, the importance of defining functions based on requirements will be discussed, to later show the hierarchy and composition of the CanSat systems and subsystems. Finally, in this first section, the final concept of the CanSat will be shown.\\

\noindent Following this, the detailed design of all the CanSat systems and subsystems will be fully presented, showing the criteria, methods and tools used for the conception of each one individually. Electronic and data management systems will be addressed, followed by structural and energy systems, as well as CanSat's own systems.\\

\noindent Subsequently, the integration and verification tests of the systems and subsystems will be shown, thus being able to appreciate the results obtained in each test. When reporting the tests, the requirements to be verified, the procedure used, the expected results and the results obtained are mentioned in each one.\\

\noindent The analysis of the results generated during the tests carried out on the CanSat in its entirety will also be addressed. These tests correspond to the launches carried out with a drone and at a much lower height than specified in the competition in its 2019 edition. This section will disply the graphs generated from the data obtained from the CanSat during the launches. Finally, the conclusions will be presented.\\

